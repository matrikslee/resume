% !TEX TS-program = xelatex
% !TEX encoding = UTF-8 Unicode
% !Mode:: "TeX:UTF-8"

\documentclass{resume}
\usepackage{zh_CN-Adobefonts_external} % Simplified Chinese Support using external fonts (./fonts/zh_CN-Adobe/)
%\usepackage{zh_CN-Adobefonts_internal} % Simplified Chinese Support using system fonts
\usepackage{linespacing_fix} % disable extra space before next section
\usepackage{cite}
\setlength{\baselineskip}{21.7pt}

\begin{document}
\pagenumbering{gobble} % suppress displaying page number

\name{李成}

\basicInfo{
  \email{im.lechain@gmail.com} \textperiodcentered\ 
  \phone{(+86) 133-7470-7978} \textperiodcentered\ 
  \github[github.com/matrikslee]{https://github.com/matrikslee}}

\section{\faGraduationCap\ 教育背景}
\datedsubsection{\textbf{中国海洋通大学}, 青岛}{2014/9 -- 2018/6}
\textit{学士}\ ——工程学院——自动化专业

\section{\faUsers\ 工作经历/项目经历}
\datedsubsection{\textbf{联发软件设计(深圳)有限公司} }{2019年4月 -- 2022年4月}
\role{嵌入式软件工程师}{linux kernel 下的 platform/device driver 开发}

2021~2022:ARM64 平台的 Linux 标准内核模块开发
\begin{itemize}
  \item Arm64 平台 kernel 模块标准 device driver 的开发经验,主要是 v4l2 device,
  \item 了解 dma\_buf, mmap 等 kernel 基础设施的用途及使用场景,并有一定的编码经验,
  \item 内核模块代码在 x86(x64) 平台下的 google test 单元测试环境搭建,使用 stub/mock 方法,
  \item 理解操作系统调用,有用户空间与内核空间程序交互的编码实践及调试经验,有 device driver 的用户空间设备接口以及系统 sysfs 接口的实现经验。
\end{itemize}

2019 年 4 月——2021 年:Mediatek 8K 高端 TV 芯片的 HDMI 设备驱动开发与维护
\begin{itemize}
  \item Arm64 平台的内存、寄存器和中断模型认知,linux kernel 环境的编码经验,
  \item IC 验证之后的公版驱动软件开发维护,处理内部公版测试中的驱动软件 bug,
  \item 客户项目的客户支持及 BUG 处理,并取得若干内部 Award 奖励,
  \item 熟悉HDMI 规范:TMDS/FRL 信号传输协议,Video制式标准等。
\end{itemize}

\datedsubsection{\textbf{索尼精密部件(惠州)有限公司}}{2018年7月 -- 2019年4月}
\role{嵌入式软件工程师}{工厂产线嵌入式设备及其上位机软件开发与维护}
1.参与支持新产品线的研发,
\begin{itemize}
  \item 按要求修改嵌入式设备源代码,编辑烧录运行,
  \item 收集和处理生产线设备数据,并根据数据报告针对性调整设备参数,
\end{itemize}
2.自行开发了用于生产线设备通信控制的上位机程序,用于调试与数据采集
\begin{itemize}
  \item 基于事件的串口文本异步接收,基于C\#和WinForm实现用户界面,
  \item 实现收发文本的编码转换 (ASCII,Unicode,UTF-8),以及日志存储功能。
\end{itemize}

\datedsubsection{\textbf{ 中国海洋大学}}{2015 年10月 -- 2017年8月}
\role{单片机 C 语言开发}{参赛项目:基于单片机的两轮自平衡小车}
本项目是单片机裸机程序,在单片机厂商提供的基础代码库上开发。作为软硬件综合团队项目,队友负责硬件 PCB 设计等,本人负责 C 代码的编写和维护,构建程序框架,并维护运行两年。
\begin{itemize}
  \item 传感器:陀螺仪、加速度计、电磁传感器、摄像头,
  \item 电机驱动:H 桥驱动电路实现电机的正反转和速度控制,
  \item 控制算法:PID 控制算法,对两轮平衡小车的姿态、速度以及方向的控制,
  \item 图像算法:二值图边界判定算法,寻找赛道边界,做为小车循迹的依据,
  \item 滤波算法:卡尔曼滤波器和平衡互补滤波器,对传感器采集数据滤波,过滤信号扰动。
\end{itemize}

% Reference Test
%\datedsubsection{\textbf{Paper Title\cite{zaharia2012resilient}}}{May. 2015}
%An xxx optimized for xxx\cite{verma2015large}
%\begin{itemize}
%  \item main contribution
%\end{itemize}

\section{\faCogs\ 工作技能}
% increase linespacing [parsep=0.5ex]
开发平台:
\begin{itemize}[parsep=0.5ex]
  \item 2 年单片机裸机 C 语言程序开发经验,
  \item 3 年的 Arm64 平台 Linux 内核设备驱动软件开发经验。
\end{itemize}
计算机体系及操作系统原理:
\begin{itemize}[parsep=0.5ex]
  \item 理解冯·诺伊曼存储计算模型的原理,对 x86(64) 以及 arm 指令体系架构有简单认知,
  \item 熟悉计算机体系结构,程序运行以及系统引导原理,对于内存管理、进程管理和线程调度、总线及存储模型、同步原语等概念有简单认知
  \item 大学期间曾完成对《深入理解计算机系统》(CS: APP)的自学,
  \item 2021 年完成《程序员的自我修养:链接、装载与库》大部分,并简单通读《Linux 内核完全注释》,
\end{itemize}
编程语言:
\begin{itemize}[parsep=0.5ex]
  \item 精通 C 语言程序编写,有大量 (7 年) 的编码实践,
  \item 能读写简单的汇编程序,
  \item 能编写简单的 bash/makefile 脚本,
  \item 对其他多种编程语言 (Java/Python/Rust/Lisp) 等有所涉猎,
    \item 大学本科毕业设计使用 java 实作一个基于 web 的信息管理系统
    \item 个人练习使用 python 简单编写过文本及图片爬虫
    \item 阅读《计算机程序的构造和解释》时学习了解过 lisp
\end{itemize}
算法和数据结构:
\begin{itemize}[parsep=0.5ex]
  \item 有算法竞赛背景,
    \item 高中参与信息奥林匹克竞赛,获湖南省联赛金牌,
    \item 大学参与大学生 ACM 竞赛,获山东省银牌。
  \item 掌握简单的数据结构与算法,如队列、链表的多种结构,以及贪心算法,背包算法等,
  \item 对各种高级数据结构和算法有简单认知,可以无障碍阅读大多数开源代码。
\end{itemize}

~\\
\section{\faHeartO\ 获奖情况}
\datedline{第十二届全国大学生“恩智浦”杯智能汽车竞赛 \textit{全国二等奖,山东省一等奖}}{2017 年}
\datedline{“浪潮杯”山东省第七届 ACM 大学生程序设计竞赛 \textit{银牌}}{2016年}
\datedline{中国海洋大学第七届“朗讯杯”科技实训比赛 \textit{二等奖}}{2016年}
\datedline{中国海洋大学 2015~2016 学年 \textit{科技创新奖学金}}{2016年}
\datedline{全国大学生数学建模竞赛 \textit{山东赛区二等奖}}{2016年}
\datedline{全国青少年信息学奥林匹克联赛 \textit{湖南省联赛一等奖}}{2012年}

~\\
\section{\faInfo\ 其他个人情况介绍}
% increase linespacing [parsep=0.5ex]
\begin{itemize}[parsep=0.5ex]
  \item 传感器、信号处理、自动控制及控制理论基础:本科自动化专业课,
  \item 通信协议:掌握 I2C, RS232 等简单的串行通信协议,掌握 HDMI 规范的编码及通信协议,丰富的HDMI 设备问题处理经验,
  \item 计算机网络基础:读完《TCP/IP 详解》第一卷,并在 Archlinux PC 上有一定的实践,
  \item 热爱计算机技术,广泛涉猎一些计算机技术,学习能力强,使用 Archlinux 作为日常 PC 操作系统,能简单分析及处理 linux 系统中常见的软件问题,
  \item 经常阅读 LWN.net 等 linux 开源社区文章,关注 linux kernel 的提交并有尝试了解感兴趣的修改提交。
\end{itemize}

%% Reference
%\newpage
%\bibliographystyle{IEEETran}
%\bibliography{mycite}
\end{document}
